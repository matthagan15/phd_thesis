\chapter{Introduction}
% \section{Attempt 2}
The goal of this thesis is to serve as a blueprint for creating quantum channels that can prepare thermal states of arbitrary systems. This framework was primarily created as an algorithmic process to prepare input states of the form $\rho(\beta) = \frac{e^{-\beta H}}{{\rm Tr} e^{-\beta H}}$, known as Gibbs, Boltzmann, or thermal states, for simulations on a digital, fault-tolerant quantum computer. As $\rho(\beta)$ approaches the ground state as the inverse temperature $\beta$ diverges, meaning Gibbs states serve as useful proxies for problems in which ground states dictate features of dynamics, such as the electron wavefunction in banded condensed matter systems or in chemical scenarios. One of the key features of our algorithm is the addition of only one single extra qubit outside of those needed to store the state of quantum system being simulated. This may seem like a minor technical achievement, given the existence of Gibbs samplers that also have only one extra qubit or at worst a constant number of qubits overhead, but the way in which this single qubit is utilized in our algorithm highlights the connections between our channel as an algorithmic tool to prepare states of interest and the model of thermalization that we believe the physical world may actually follow. The rest of this introduction is to provide context for how the technical results contained in later chapters of this thesis contribute to the growing interplay between Physics and Theoretical Computer Science within the realm of Quantum Computing.

The easiest way to grasp the context of this thesis is to first understand classical thermal states. In classical mechanics we typically have access to a Hamiltonian $H$ that is a function on phase space $(x, p)$ to the reals $\mathbb{R}$. We can turn this function into a probability distribution via the canonical ensemble $p_\beta (x, p) = \frac{e^{- \beta H(x, p)}}{\int dx dp e^{-\beta H(x, p)}}$, which can be thought of as the ``density'' of particles near a particular $(x,p)$ in phase space for a thermodynamically large collection of non-interacting particles under the same Hamiltonian $H$. The inverse temperature $\beta$, typically taken to be $\frac{1}{k_\beta T}$ where $k_\beta$ is Boltzmann's constant and T the temperature, plays an important role in shaping the distribution $p_\beta(x,p)$. For example, in the $\beta \to \infty$ the distribution becomes concentrated at the minimum energy points of $H(x, p)$, which correspond to points with zero momentum $p=0$ and are minimums of the potential energy $V(x)$. This shows that the problem of preparing classical thermal states somehow ``contains'' the problem of function optimization, and in fact being able to sample from classical thermal states for arbitrarily large $\beta$ allows us to approximate the partition function $\mathcal{Z} = \int e^{-\beta H(x,p)} dx dp$, which is known to be a \#P-Hard computational problem.



\section{Motivation}

This intuitive understanding breaks down at the quantum level. For starters, to observe delicate quantum effects experimentalists will typically try to isolate their desired system as much as possible from the environment, which can decohere sensitive wavefunctions. When the system is completely isolated from the environment, what does ``equilibrium'' even refer to? One of the leading answers to this question says that for large systems, sufficiently ``random looking'' Hamiltonians can appear to be in thermal equilibrium to local observables. A large enough, random enough system can act as it's own bath to small subsystems. The codification of these conditions is known as the Eigenstate Thermalization Hypothesis (ETH), which is yet to be proven for all possible systems (hence the name Hypothesis).

The picture of thermalization that we address is an intermediate model of the two previously mentioned, where instead of modeling infinite baths or having no baths at all we instead consider a very tiny environment that is consistently refreshed. This model, known as the Repeated Interactions (RI) model, is appealing for many different reasons. For starters, since the environment is small enough we can often compute the effects on the system directly. So far, physicists have mostly studied this model for small systems in which these interactions can be explicitly modeled, such as 2 or 3 level systems. Oftentimes the interaction model between these small systems and the environment is chosen to lead to a thermal state on the system.

We extend the Repeated Interactions model to arbitrary non-degenerate systems via a randomized interaction. This not only extends the validity of the Repeated Interactions model greatly, thus providing a physically plausible model for thermalization, but also provides an algorithm for preparing thermal states on digital quantum computers. We are able to prove both correctness, meaning the output of the algorithm is $\epsilon$ close to the thermal state, and bound the runtime in the weak coupling limit. This runtime bound is interesting in it's own right as bounding the runtime of the algorithm in the ground state limit is difficult, however in our scenario the runtime bound becomes completely explicit. 

Although the resulting quantum channel we develop almost directly resembles the Repeated Interactions framework, the route we took was inspired entirely by the classical sampling algorithm called Hamiltonian Monte Carlo (HMC). HMC takes the original Metropolis-Hastings algorithm, which involves a filtered random walk over the state space, and modifies the random walk steps to instead simulate time evolution under Hamiltonian dynamics. This modification drastically improved the original algorithm's scaling with respect to the dimension of the problem and offered less correlated samples empirically. This thesis can be viewed as an effort to extend this algorithm, which is heavily inspired by physics, to the realm of quantum computing. 

This thesis is organized as follows. The rest of this chapter will be devoted to preliminary discussion on relevant quantum computing topics, namely how to implement time evolution operators for Pauli sum Hamiltonians. The next chapter is devoted to improvements in basic product formula techniques via a composite or partially randomized compilation scheme. We then utilize these results in Chapter \ref{ch:tsp} to develop our quantum algorithm for preparing thermal states. To complement our analytic discussions we provide extensive numeric evidence in support of our routines. 

\section{Introduction to Product Formulae}
One of the key themes in this thesis is that time dynamics can be used to generate thermal states. In this section we introduce one of the most straightforward methods for implementing time independent Hamiltonian evolution on a quantum computer. We will not introduce basic quantum computing preliminaries, such as universality and various gate sets, but we will take strides to reduce the algorithms to primitives that are as simple as possible. 

Quantum mechanics starts with the notion of a Hilbert space. Oftentimes problems can become much clearer once one can clearly identify and manipulate the Hilbert space under investigation. We will work with relatively straightforward Hilbert spaces in this thesis, those consisting of $n$ qubits $\mathcal{H} = \CC^{2} \otimes \CC^2 \otimes \ldots \otimes \CC^2 = \CC^{2^n}$. Watrous prefers the term ``Complex Euclidean Spaces'' for these finite dimensional Hilbert spaces. A state $\ket{\psi} \in \mathcal{H}$ is an $L_2$ normalized vector and given a Hamiltonian $H$ that dictates the dynamics of the system, the time evolution is given by the time independent Schrodinger equation
\begin{equation}
    i \hbar \frac{\partial}{\partial t} \ket{\psi(t)} = H(t) \ket{\psi(t)}.
\end{equation}
For the rest of this thesis $\hbar = 1$. 

Our use of qubit Hilbert spaces gives us access to the Pauli operators $I_i, X_i, Y_i, Z_i$ for each qubit $i$, which we can then string together to form a Pauli string, here is an example on 4 qubits $P = X_1 \otimes I_2 \otimes Y_3 \otimes Z_4$. Typically we will drop the tensor product symbol $\otimes$ and drop the identity factors, so the previous example will be written as $P = X_1 Y_3 Z_4$. A useful fact about Pauli operators is that they can be exponentiated fairly easily due to the identity $e^{i \theta P} = \cos(\theta) I + i \sin(\theta) P$. We now will show how these rotations can be implemented given only single qubit rotations and CNOT gates. Here is a basic example of how to do a two qubit rotation $e^{i \theta Z \otimes Z}$.
\begin{table}[h]
    \[
    \begin{array}{c c c} 
        \Qcircuit @C=1em @R=.7em {
            & \ctrl{2} & \qw      & \qw                & \qw      & \ctrl{2} & \qw \\
            & \qw      & \ctrl{1} & \qw                & \ctrl{1} & \qw      & \qw\\
            & \targ    & \targ    & \gate{R_Z(2 \theta)} & \targ    & \targ    & \qw
        } &  = & \Qcircuit @C=1em @R = 0.7em {
             & \gate{e^{i \theta Z_1 Z_2}} & \qw
        }
    \end{array}
    \]
    \caption{$Z_1 Z_2$ rotation.}
\end{table}
This is a slightly abstracted circuit, as arbitrary angle $Z$ rotations are not typical circuits that real quantum computers can do. To further compile this, one would need to decompose the single qubit $Z$ rotation into some universal gate set, typically Clifford + T, using an algorithm such as the Ross-Sellinger single qubit rotation algorithm. For this thesis we will assume access to single qubit rotations and CNOT gates as our universal gate set. 

To extend this to arbitary $n$ qubit Paulis with possibly $X$ or $Y$ Paulis as well we can repeat the same circuit but with a change of basis. With the basic circuit identities that $X = H Z H$ and $Y = S Z S^\dagger$. We can then change basis from the computation basis states to those of the Pauli we are interested in, do the computation in the $Z$ basis, and then change back to the computation basis. For instance, here is how we could do the rotation $e^{i X_1 Y_2 Z_3 \theta}$
\begin{table}[h]
\[
\begin{array}{c}
    \Qcircuit @C=1em @R=0.8em {
        & \gate{H} & \ctrl{3} & \qw & \qw & \qw & \qw & \qw & \ctrl{3} & \gate{H} \\
        & \gate{S^\dagger} & \qw & \ctrl{2} & \qw & \qw & \qw & \ctrl{2} & \qw & \gate{S} \\
        & \gate{I} &\qw & \qw & \ctrl{1} & \qw & \ctrl{1} & \qw & \qw & \gate{I} \\
        \lstick{\ket{0}} & \qw & \targ & \targ & \targ & \gate{R_Z(2\theta)} & \targ & \targ & \targ & \qw 
    }
\end{array}
\]
\caption{Mixed Pauli rotations}
\end{table}
\todo{Lemmize this}

We will now collect these basic observations, along with an optimization that eliminates the ancilla qubit, in the following lemma.
\begin{lemma} \label{lem:pauli_rotations}
    Let $P$ be an $n$ qubit Pauli operator and $U_P$ the change of basis unitary from a product of $Z$ operators, i.e 
    \begin{equation}
        P = U_P \left(Z_1^{p_1} Z_2^{p_2} \ldots Z_n^{p_n} \right) U_P^\dagger,
    \end{equation}
    where each power $p_i$ is 0 or 1. The support of $P$, or the qubits in which $P$ acts nontrivially on, is denoted ${\rm supp}(P)$ and is the set of nonzero $p_i$. We denote the cardinality $|{\rm supp}(P)| = w$, as it is typically referred to as the weight of $P$, and let $s_w$ denote the largest index in supp$(P)$. The time evolution operator $e^{i P t}$ can be performed with $2(w-1)$ CNOT gates and one single qubit $Z$ rotation via the following identity
    \begin{equation}
        e^{i P t} = U_P \left(\prod_{i \in {\rm supp}(P) \setminus s_w} {\rm CNOT}(i, s_w) R_{Z_{s_w}}(t) \prod_{i \in {\rm supp}(P) \setminus s_w} {\rm CNOT}(i, s_w) \right) U_P^\dagger.
    \end{equation}
    This construction uses no additional ancilla qubits.
\end{lemma}
\begin{proof}
    We first note that 
    \begin{equation}
        e^{i P t} = \sum_{k = 0}^\infty \frac{(i t)^k}{k!} P^k = \sum_{k = 0}^\infty \frac{(i t)^k}{k!} (U_P \prod_i Z_i^{p_i} U_P^\dagger)^k = U_P \left(\sum_{k = 0}^\infty \frac{(i t)^k}{k!} ( \prod_i Z_i^{p_i})^k \right) U_P^\dagger = U_P e^{i \prod_i Z_i^{p_i} t} U_P^\dagger. \label{eq:change_basis_of_exponential}
    \end{equation}
    This reduces our problem from arbitrary Pauli rotations to just Pauli $Z$ rotations. We now need to show that the CNOT sequence and single qubit $Z$ rotation perform arbitrary tensor product $Z$ rotations. Let CNOT$(i,j)$ be a controlled NOT gate with control qubit $i$ and target qubit $j$. Let $\ket{s} = \ket{s_1} \ket{s_2} \ldots \ket{s_w}$ denote a basis state for the qubits that are in the support of $P$. As CNOT$(1,2) \ket{a}\ket{b} = \ket{a} \ket{a \oplus b}$, where $a \oplus b$ is the addition of $a$ and $b$ mod 2, we have that the product
    \begin{equation}
        \prod_{i \in {\rm supp}(P) \setminus s_w} {\rm CNOT}(i, s_w) \ket{s} = \ket{s_1} \ldots \ket{s_{w-1}} \ket{s_1 \oplus s_2 \oplus \ldots \oplus s_w}
    \end{equation}
    stores the parity of the entire basis state in the last qubit. Applying a $Z$ rotation on the last qubit then yields the following
    \begin{align}
        R_{Z_{s_w}}(t) \ket{s_1} \ldots \ket{s_{w-1}} &= \ket{s_1} \ldots \ket{s_{w- 1}} \left(R_Z(t) \ket{s_1 \oplus \ldots \oplus s_w} \right) \\
        &= e^{i \bigoplus_{i = 1}^w s_i t} \ket{s_1} \ldots \ket{s_{w- 1}} \ket{s_1 \oplus \ldots \oplus s_w}.
    \end{align}
    Performing the same sequence of CNOT gates in reverse then ``uncomputes'' the parity
    \begin{align}
        \prod_{i \in {\rm supp}(P) \setminus s_w} {\rm CNOT}(i, s_w) e^{i \bigoplus_{i = 1}^w s_i t} \ket{s_1} \ldots \ket{s_{w- 1}} \ket{s_1 \oplus \ldots \oplus s_w} &= e^{i \bigoplus_{i = 1}^w s_i t} \ket{s_1} \ldots \ket{s_{w- 1}} \ket{s_w}.
    \end{align}
    As the circuit performs as desired on an arbitrary basis state, by linearity it extends to an arbitrary superposition. Combined with the fact that the basis only needs to be changed once, as show in Eq. \eqref{eq:change_basis_of_exponential}, we are done.
\end{proof}

Lemma \ref{lem:pauli_rotations} serves as the basis for a class of simulation algorithms known as Pauli formulas.
We will introduce and prove two techniques for time-independent Hamiltonian simulation in this section, namely Trotter formulas and the randomized channel QDRIFT, as they serve as the basis for the optimizations that we introduce later in Chapter \ref{ch:composite_sims}.
The first order Trotter formula was the first simulation algorithm and was developed by Lloyd \cite{lloyd1996universal}. 
The advantages of this algorithm is that it is straightforward, easy to implement, and requires no ancilla qubits. 
The downside is that analytically our bounds on the runtime of the algorithm scale as $O\left( L \frac{t^2}{\epsilon}\right)$ for the first order formula, which is significantly worse than the lower bound of $\Omega\left(t + \log \frac{1}{\epsilon} \right)$. 
This spurred the development of more sophisticated techniques for the Hamiltonian simulation problem, many of which require a different input model, namely a block-encoding, for the Hamiltonian than the Pauli decomposition we consider, leading to asymptotically optimal algorithms \cite{low2019hamiltonian} that utilize only a single additional ancilla qubit. However, many practitioners observed that in practice Trotter algorithms tended to drastically outperform their worst case analyses. The analysis of product formulas was improved to account for the commutation structure of the Hamiltonian in \cite{childs2021theory} and was shown to be superior to analytically ``optimal'' techniques for certain Hamiltonians. Recently improved analysis of Trotter formulas showed that entanglement can improve these error estimates, so progress is still being made in undestanding the performance of Trotter nearly 30 years later. For now, we will present a complete proof of the first order Trotter formula and provide references for the higher order formulas.

Before we present the first order Trotter theorem we will need to define some channels first. Let $\mathcal{U}(t)$ denote the ideal time evolution channel with respect to a hamiltonian $H = \sum_{i = 1}^L h_i H_i$
\begin{equation}
    \mathcal{U}(t) \coloneqq U(t) \rho U(t)^\dagger = e^{-i H t} \rho e^{+i H t}.
\end{equation}
The first order Trotter formula is given by the following unitary
\begin{equation}
    U_{\rm TS}^{(1)}(t) \coloneqq e^{i h_1 H_1 t}e^{i h_2 H_2 t} \ldots e^{i h_L H_L t} = \prod_{i = 1}^L e^{i h_i H_i t}.
\end{equation}
This first order expression can be used to build higher order formulas, which we denote $U_{\rm TS}^{(2k)}$ for a $2k$\ts{th} order formula unitary and $\mathcal{U}_{\rm TS}^{(2k)}$ for the channel. The higher order expressions along with their runtime costs and errors are presented in detail in Section \ref{sec:composite_prelim}.


\begin{theorem}[First Order Trotter]
    Given a Hamiltonian $H = \sum_{i = 1}^L h_i H_i$, there exists an $r$ such that the first order Trotter formula $U_{\rm TS}^{(1)}(t)$ is an $\epsilon$ approximation to the ideal time evolution
    \begin{equation}
        \diamondnorm{\mathcal{U}(t) - \mathcal{U}_{\rm TS}^{(1)}(t/r)^{\circ r}} \le \epsilon
    \end{equation}
    via the diamond distance. This $r$ can be bounded by 
    \begin{equation}
        r \le
    \end{equation}
    leading to an overall cost in terms of the number of hamiltonian evolution gates as 
    \begin{equation}
        C_{\rm TS(1)}(H, t, \epsilon) \le .
    \end{equation}
\end{theorem}
Before we prove this theorem we will need a minor lemma that allows us to convert bounds on the spectral norm approximation to the diamond distance. We separate this result into its own lemma to make use of it for higher order Trotter formulas as well.
\begin{lemma}
    Given two unitary channels $\mathcal{U}(\rho) = U \rho U^\dagger$ and $\mathcal{V}(\rho) = V \rho V^\dagger$ the diamond distance is at most twice the spectral norm distance between $U$ and $V$
    \begin{equation}
        \diamondnorm{\mathcal{U} - \mathcal{V}} \le 2 \norm{U - V}.
    \end{equation}
\end{lemma}
\begin{proof}
    This proof follows from standard matrix norm manipulations and involves the use of the unitary invariance of the Sch\"{a}tten norms, invariance under the adjoint, $\norm{X}_p = \norm{X^\dagger}_p$, and one use of the H\"{o}lder inequality
    \begin{align}
        &\diamondnorm{\mathcal{U} - \mathcal{V}} \coloneqq \norm{\parens{\mathcal{U} - \mathcal{V}}\otimes \openone}_1 \label{eq:diamond_to_spectral_start} \\
        =& \max_{\rho : \norm{\rho}_1 \leq 1} \norm{\parens{U \otimes \openone} \rho \parens{U^\dagger \otimes \openone} - \parens{V \otimes \openone} \rho \parens{V^\dagger \otimes \openone} }_1 \\
        \leq& \max_{\rho : \norm{\rho}_1 \leq 1} \norm{\parens{U \otimes \openone} \rho \parens{U^\dagger \otimes \openone} - \parens{U \otimes \openone} \rho \parens{V^\dagger \otimes \openone} }_1 \\
        ~& +\max_{\rho : \norm{\rho}_1 \leq 1} \norm{\parens{U \otimes \openone} \rho \parens{V^\dagger \otimes \openone}  - \parens{V \otimes \openone} \rho \parens{V^\dagger \otimes \openone} }_1 \nonumber \\
        =& \max_{\rho : \norm{\rho}_1 \leq 1} \norm{\rho \parens{U^\dagger - V^\dagger}\otimes \openone}_1 + \max_{\rho : \norm{\rho}_1 \leq 1} \norm{\parens{U - V}\otimes \openone \rho}_1 \\
        \leq & 2 \norm{U - V}_\infty \max_{\rho : \norm{\rho}_1 \leq 1} \norm{\rho}_1 \\
        = & 2 \norm{U - V}. \label{eq:TS_intermediate_1}
    \end{align}
\end{proof}
Now we turn to our proof of the first order Trotter formula.
\begin{proof}
    Consider the Taylor expansion of the exact single step evolution
    \begin{equation}
        e^{i H t/r} = \identity + \frac{i t}{r} \sum_i h_i H_i + R(t / r)
    \end{equation}
    compared to the first order Trotter expression
    \begin{equation}
        \prod_{i = 1}^L e^{i h_i H_i t / r} = \prod_{i = 1}^L \left( \identity + \frac{i t}{r} h_i H_i + R_{H_i}(t/r) \right) = \identity = \frac{i t}{r} \sum_{i} h_i H_i + R_{\rm TS}(t/r).
    \end{equation}
    The remainder term $R$ denotes the ideal remainder, $R_{H_i}$ the remainder for the $H_i$ factor, and $R_{\rm TS}$ the overall Trotter-Suzuki remainder. The mean value variant of Taylor's remainder theorem guarantees that for all $t/r \in [0, t]$ there exists a special value $t_\star$ such that the above equalities hold and the value for the remainders are given by
    \begin{align}
        R_{\rm TS}(t/r) &= \frac{(i t_\star)^2}{2!} H^2 \\
        R_{H_i}(t/r) &= \frac{(i t_\star' h_i)^2}{2!} H_i^2 .
    \end{align}

    First lets prove it for two terms. The Trotter remainder contributes as
    \begin{align}
        e^{i h_1 H_1 t} e^{i h_2 H_2 t} &= \left( \identity + i t h_1 H_1 + \frac{- t_\star^2 h_1^2}{2!} H_1^2 \right)\left( \identity + i t h_2 H_2 + \frac{- t_\star'^2 h_2^2}{2!} H_2^2 \right) \\
        &= \identity + i t (h_1 H_1 + h_2 H_2) - \frac{t_\star^2}{2} h_1^2 H_1^2 - \frac{t_\star'^2 }{2!} h_2^2 H_2^2 - t^2 h_1 h_2 H_1 H_2 + \bigo{t^3},
    \end{align}
    whereas the ideal remainder error is
    \begin{align}
        e^{i H t} &= \identity + i H t - \frac{t_\star''^2}{2!} (h_1 H_1 + h_2 H_2)^2 \\
        &= \identity + i H t - \frac{t_\star''^2}{2!} (h_1^2 H_1^2 + h_1 h_2(H_1 H_2 + H_2 H_1) + h_2^2 H_2^2).
    \end{align}
    This gives the difference as
    \begin{align}
        &\norm{U(t) - U_{\rm TS}(t)} \\
        &\le \norm{\frac{1}{2} h_1^2 H_1^2 (t_\star^2 - t_\star''^2) +  \frac{1}{2} h_2^2 H_2^2 (t_\star^2 - t_\star'^2) +  \frac{1}{2} t_\star''^2 h_1 h_2 (H_1 H_2 + H_2 H_1) - t^2 h_1 h_2 H_1 H_2} + \bigo{t^3}.
    \end{align}
    Setting $t_\star = t_\star' = t_\star'' = t$ yields
    \begin{equation}
        \norm{U(t) - U_{\rm TS}(t)} \le \frac{t^2}{2} h_1 h_2 \norm{[H_1, H_2]} + \bigo{t^3}. \label{eq:first_order_trotter_proof_1}
    \end{equation}

    In order to bound the error we will repeat our unitary $r$ times for time steps of $t/r$, which yields
    \begin{equation}
        \norm{U(t) - U_{\rm TS}(t)} = \norm{U(t/r)^r - U_{\rm TS}(t/r)^r} \le r \norm{U(t/r) - U_{\rm TS}(t/r)}.
    \end{equation}
    Now by ignoring the contribution of the $\bigo{t^3}$ term in Eq. \ref{eq:first_order_trotter_proof_1} and plugging the above into Eq. \ref{eq:first_order_trotter_proof_1} we have that 
    \begin{equation}
        \norm{U(t) - U_{\rm TS}(t)} \le \frac{t^2}{2 r} h_1 h_2 \norm{[H_1, H_2]},
    \end{equation}
    which we can then choose $r \ge \frac{t^2}{\epsilon}h_1 h_2 \norm{[H_1, H_2]}$ is sufficient to guarantee $\norm{U(t) - U_{\rm TS}(t)} \le \epsilon / 2$.
\end{proof}

$$\ket{a}\bra{b} vs. \ketbra{a}{b} vs. |a\rangle\langle b|$$

The next thing we have to discuss is the QDrift Channel. This channel is a randomized product formula, meaning that instead of visiting the terms in the Hamiltonian one by one, we will randomly choose a term from the Hamiltonian and simulate it. By repeating this process we can approximate the overall time evolution channel. As we are implementing a probabilistic application of a unitary time evolution we need to represent the algorithm as a quantum channel acting on density matrices
\begin{equation}
    \mathcal{U}_{\rm QD}(\rho; t) \coloneqq \sum_{i = 1}^L \frac{h_i}{\norm{h}} e^{-i H_i \tau} \rho e^{+i H_i \tau}.
\end{equation}

\begin{itemize}
    \item What exactly is this thesis doing? The composite simulation section shows how to split a hamiltonian up into two pieces so that you can tailor the simulation algorithm for the hamiltonian of interest. 
    \item The next section shows how to turn an ideal time evolution channel into a thermalizing channel. 
    \item The introduction should aim to motivate these two topics. There should not be any theorems or proofs in the introduction.
\end{itemize}
