\chapter{Introduction}
\setstretch{2.0}
One of the most challenging issues facing physicists, chemists, and materials scientists throughout the 20th and 21st centuries is the problem of scale. Throughout a typical undergraduate program in which one learns quantum mechanics the systems being studied are typically toy models, small instances that can be solved and computed by hand. A simple example of this phenomenon is that When studying realistic systems it quickly becomes impossible to completely solve the equations of motion exactly by hand. This is best captured by the philosophy ``More is Different" by Nobel Laureate Paul Anderson. Different fields have different techniques for approaching this complexity, quantum chemists rely on orbitals and mean-field theories, condensed matter studies invariants, typically topological in nature, across very large scales, and high energy physicists utilize lattice based methods. The ``More is Different'' philosophy argues that the techniques and tools developed to solve these problems are theoretically just as ``valid'' as those traditionally developed by physicists to study the microscopic laws of nature.

The real promise of quantum computers is they offer a way to drastically improve the numerical simulation of quantum systems at scale. Although quantum systems can be studied using classical computers the algorithms to do so tend to require an exponential increase in the resources needed with the number of particles being simulated. For example, one can classically simulate a single trajectory in a path integral fairly efficiently if the Hamiltonian is efficiently computable. However one typically encounters the sign problem where the phase of each path must be computed and an exponential number of paths must be sampled from to accurately estimate a quantity of interest. Quantum computers avoid this by directly simulating the system of interest on a quantum device that can be controlled with exquisite precision. This then allows us to evolve the system over time and to estimate quantities of interest directly. 

This thesis is centered around the development of algorithms to implement time evolution and cooling processes on a fault-tolerant digital quantum computer. 

To simplify the computational model we will assume access to a noise and decoherence free quantum computer, typically called a fault-tolerant device. This allows us to abstract away any specific hardware and error-correction platform and focus solely on the computational task of guaranteeing an accurate quantum simulation. The main problems encountered in simulating quantum systems are preparing initial states and performing the time evolution operator $U = e^{i H t}$ given access to $H$. This thesis develops and studies two new algorithms for tackling these problems, providing extensive analytic and numeric evidence that investigates their performance. The rest of this section will develop the tools needed for these topics in technical detail. 